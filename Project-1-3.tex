% Options for packages loaded elsewhere
\PassOptionsToPackage{unicode}{hyperref}
\PassOptionsToPackage{hyphens}{url}
%
\documentclass[
  12pt,
]{article}
\usepackage{amsmath,amssymb}
\usepackage{lmodern}
\usepackage{iftex}
\ifPDFTeX
  \usepackage[T1]{fontenc}
  \usepackage[utf8]{inputenc}
  \usepackage{textcomp} % provide euro and other symbols
\else % if luatex or xetex
  \usepackage{unicode-math}
  \defaultfontfeatures{Scale=MatchLowercase}
  \defaultfontfeatures[\rmfamily]{Ligatures=TeX,Scale=1}
\fi
% Use upquote if available, for straight quotes in verbatim environments
\IfFileExists{upquote.sty}{\usepackage{upquote}}{}
\IfFileExists{microtype.sty}{% use microtype if available
  \usepackage[]{microtype}
  \UseMicrotypeSet[protrusion]{basicmath} % disable protrusion for tt fonts
}{}
\makeatletter
\@ifundefined{KOMAClassName}{% if non-KOMA class
  \IfFileExists{parskip.sty}{%
    \usepackage{parskip}
  }{% else
    \setlength{\parindent}{0pt}
    \setlength{\parskip}{6pt plus 2pt minus 1pt}}
}{% if KOMA class
  \KOMAoptions{parskip=half}}
\makeatother
\usepackage{xcolor}
\usepackage[margin=1in]{geometry}
\usepackage{color}
\usepackage{fancyvrb}
\newcommand{\VerbBar}{|}
\newcommand{\VERB}{\Verb[commandchars=\\\{\}]}
\DefineVerbatimEnvironment{Highlighting}{Verbatim}{commandchars=\\\{\}}
% Add ',fontsize=\small' for more characters per line
\usepackage{framed}
\definecolor{shadecolor}{RGB}{248,248,248}
\newenvironment{Shaded}{\begin{snugshade}}{\end{snugshade}}
\newcommand{\AlertTok}[1]{\textcolor[rgb]{0.94,0.16,0.16}{#1}}
\newcommand{\AnnotationTok}[1]{\textcolor[rgb]{0.56,0.35,0.01}{\textbf{\textit{#1}}}}
\newcommand{\AttributeTok}[1]{\textcolor[rgb]{0.77,0.63,0.00}{#1}}
\newcommand{\BaseNTok}[1]{\textcolor[rgb]{0.00,0.00,0.81}{#1}}
\newcommand{\BuiltInTok}[1]{#1}
\newcommand{\CharTok}[1]{\textcolor[rgb]{0.31,0.60,0.02}{#1}}
\newcommand{\CommentTok}[1]{\textcolor[rgb]{0.56,0.35,0.01}{\textit{#1}}}
\newcommand{\CommentVarTok}[1]{\textcolor[rgb]{0.56,0.35,0.01}{\textbf{\textit{#1}}}}
\newcommand{\ConstantTok}[1]{\textcolor[rgb]{0.00,0.00,0.00}{#1}}
\newcommand{\ControlFlowTok}[1]{\textcolor[rgb]{0.13,0.29,0.53}{\textbf{#1}}}
\newcommand{\DataTypeTok}[1]{\textcolor[rgb]{0.13,0.29,0.53}{#1}}
\newcommand{\DecValTok}[1]{\textcolor[rgb]{0.00,0.00,0.81}{#1}}
\newcommand{\DocumentationTok}[1]{\textcolor[rgb]{0.56,0.35,0.01}{\textbf{\textit{#1}}}}
\newcommand{\ErrorTok}[1]{\textcolor[rgb]{0.64,0.00,0.00}{\textbf{#1}}}
\newcommand{\ExtensionTok}[1]{#1}
\newcommand{\FloatTok}[1]{\textcolor[rgb]{0.00,0.00,0.81}{#1}}
\newcommand{\FunctionTok}[1]{\textcolor[rgb]{0.00,0.00,0.00}{#1}}
\newcommand{\ImportTok}[1]{#1}
\newcommand{\InformationTok}[1]{\textcolor[rgb]{0.56,0.35,0.01}{\textbf{\textit{#1}}}}
\newcommand{\KeywordTok}[1]{\textcolor[rgb]{0.13,0.29,0.53}{\textbf{#1}}}
\newcommand{\NormalTok}[1]{#1}
\newcommand{\OperatorTok}[1]{\textcolor[rgb]{0.81,0.36,0.00}{\textbf{#1}}}
\newcommand{\OtherTok}[1]{\textcolor[rgb]{0.56,0.35,0.01}{#1}}
\newcommand{\PreprocessorTok}[1]{\textcolor[rgb]{0.56,0.35,0.01}{\textit{#1}}}
\newcommand{\RegionMarkerTok}[1]{#1}
\newcommand{\SpecialCharTok}[1]{\textcolor[rgb]{0.00,0.00,0.00}{#1}}
\newcommand{\SpecialStringTok}[1]{\textcolor[rgb]{0.31,0.60,0.02}{#1}}
\newcommand{\StringTok}[1]{\textcolor[rgb]{0.31,0.60,0.02}{#1}}
\newcommand{\VariableTok}[1]{\textcolor[rgb]{0.00,0.00,0.00}{#1}}
\newcommand{\VerbatimStringTok}[1]{\textcolor[rgb]{0.31,0.60,0.02}{#1}}
\newcommand{\WarningTok}[1]{\textcolor[rgb]{0.56,0.35,0.01}{\textbf{\textit{#1}}}}
\usepackage{graphicx}
\makeatletter
\def\maxwidth{\ifdim\Gin@nat@width>\linewidth\linewidth\else\Gin@nat@width\fi}
\def\maxheight{\ifdim\Gin@nat@height>\textheight\textheight\else\Gin@nat@height\fi}
\makeatother
% Scale images if necessary, so that they will not overflow the page
% margins by default, and it is still possible to overwrite the defaults
% using explicit options in \includegraphics[width, height, ...]{}
\setkeys{Gin}{width=\maxwidth,height=\maxheight,keepaspectratio}
% Set default figure placement to htbp
\makeatletter
\def\fps@figure{htbp}
\makeatother
\setlength{\emergencystretch}{3em} % prevent overfull lines
\providecommand{\tightlist}{%
  \setlength{\itemsep}{0pt}\setlength{\parskip}{0pt}}
\setcounter{secnumdepth}{-\maxdimen} % remove section numbering
\ifLuaTeX
  \usepackage{selnolig}  % disable illegal ligatures
\fi
\IfFileExists{bookmark.sty}{\usepackage{bookmark}}{\usepackage{hyperref}}
\IfFileExists{xurl.sty}{\usepackage{xurl}}{} % add URL line breaks if available
\urlstyle{same} % disable monospaced font for URLs
\hypersetup{
  pdftitle={Project 1},
  pdfauthor={Chloe Chen \& Kangrui Liu},
  hidelinks,
  pdfcreator={LaTeX via pandoc}}

\title{Project 1}
\author{Chloe Chen \& Kangrui Liu}
\date{9/5/2023}

\begin{document}
\maketitle

\hypertarget{use-in-package.}{%
\section{\texorpdfstring{1. Use \texttt{hprice} in \texttt{faraway}
package.}{1. Use  in  package.}}\label{use-in-package.}}

\hypertarget{what-are-the-mean-and-the-variance-of-homeprice-what-do-they-mean}{%
\subsection{1) What are the mean and the variance of homeprice? What do
they
mean?}\label{what-are-the-mean-and-the-variance-of-homeprice-what-do-they-mean}}

\begin{Shaded}
\begin{Highlighting}[]
\FunctionTok{library}\NormalTok{(}\StringTok{"faraway"}\NormalTok{)}
\FunctionTok{data}\NormalTok{(}\StringTok{"hprice"}\NormalTok{)}
\FunctionTok{dim}\NormalTok{(hprice)}
\end{Highlighting}
\end{Shaded}

\begin{verbatim}
## [1] 324   8
\end{verbatim}

\begin{Shaded}
\begin{Highlighting}[]
\NormalTok{hprice}\SpecialCharTok{$}\NormalTok{homeprice}\OtherTok{\textless{}{-}}\FunctionTok{exp}\NormalTok{(hprice}\SpecialCharTok{$}\NormalTok{narsp)}\SpecialCharTok{*}\DecValTok{1000}
\FunctionTok{mean}\NormalTok{(hprice}\SpecialCharTok{$}\NormalTok{homeprice)}
\end{Highlighting}
\end{Shaded}

\begin{verbatim}
## [1] 94411.42
\end{verbatim}

\begin{Shaded}
\begin{Highlighting}[]
\FunctionTok{var}\NormalTok{(hprice}\SpecialCharTok{$}\NormalTok{homeprice)}
\end{Highlighting}
\end{Shaded}

\begin{verbatim}
## [1] 1583110349
\end{verbatim}

\begin{itemize}
\tightlist
\item
  The mean is \(94411.42\) and the variance is \(1583110349\).\\
\end{itemize}

\hypertarget{construct-a-95-confidence-interval-of-the-average-homeprice.-what-does-the-confidence-interval-imply}{%
\subsection{2) Construct a 95\% confidence interval of the average
homeprice. What does the confidence interval
imply?}\label{construct-a-95-confidence-interval-of-the-average-homeprice.-what-does-the-confidence-interval-imply}}

\begin{Shaded}
\begin{Highlighting}[]
\NormalTok{n}\OtherTok{\textless{}{-}}\FunctionTok{dim}\NormalTok{(hprice)[[}\DecValTok{1}\NormalTok{]]}
\NormalTok{samvar}\OtherTok{\textless{}{-}}\FunctionTok{var}\NormalTok{(hprice}\SpecialCharTok{$}\NormalTok{homeprice)}\SpecialCharTok{/}\NormalTok{(n}\DecValTok{{-}1}\NormalTok{)}
\NormalTok{samvar}
\end{Highlighting}
\end{Shaded}

\begin{verbatim}
## [1] 4901270
\end{verbatim}

\begin{Shaded}
\begin{Highlighting}[]
\NormalTok{t.score}\OtherTok{\textless{}{-}}\FunctionTok{qt}\NormalTok{(}\AttributeTok{p=}\NormalTok{.}\DecValTok{05}\SpecialCharTok{/}\DecValTok{2}\NormalTok{, }\AttributeTok{df=}\NormalTok{n}\DecValTok{{-}1}\NormalTok{, }\AttributeTok{lower.tail=}\NormalTok{F)}
\NormalTok{t.score}
\end{Highlighting}
\end{Shaded}

\begin{verbatim}
## [1] 1.967336
\end{verbatim}

\begin{Shaded}
\begin{Highlighting}[]
\NormalTok{lowCI }\OtherTok{\textless{}{-}} \FunctionTok{mean}\NormalTok{(hprice}\SpecialCharTok{$}\NormalTok{homeprice)}\SpecialCharTok{{-}}\NormalTok{t.score}\SpecialCharTok{*}\FunctionTok{sqrt}\NormalTok{(samvar)}
\NormalTok{upCI }\OtherTok{\textless{}{-}} \FunctionTok{mean}\NormalTok{(hprice}\SpecialCharTok{$}\NormalTok{homeprice)}\SpecialCharTok{+}\NormalTok{t.score}\SpecialCharTok{*}\FunctionTok{sqrt}\NormalTok{(samvar)}
\FunctionTok{print}\NormalTok{(}\FunctionTok{c}\NormalTok{(lowCI,upCI))}
\end{Highlighting}
\end{Shaded}

\begin{verbatim}
## [1] 90055.97 98766.87
\end{verbatim}

\begin{Shaded}
\begin{Highlighting}[]
\FunctionTok{t.test}\NormalTok{(hprice}\SpecialCharTok{$}\NormalTok{homeprice, }\AttributeTok{conf.level =} \FloatTok{0.95}\NormalTok{)}
\end{Highlighting}
\end{Shaded}

\begin{verbatim}
## 
##  One Sample t-test
## 
## data:  hprice$homeprice
## t = 42.711, df = 323, p-value < 2.2e-16
## alternative hypothesis: true mean is not equal to 0
## 95 percent confidence interval:
##  90062.70 98760.14
## sample estimates:
## mean of x 
##  94411.42
\end{verbatim}

\hypertarget{estimate-the-average-homeprice-by-whether-the-mas-was-adjacent-to-a-coastline-noted-in-ajwtr-and-the-standard-errors.}{%
\subsection{3) Estimate the average homeprice by whether the MAS was
adjacent to a coastline, noted in ajwtr, and the standard
errors.}\label{estimate-the-average-homeprice-by-whether-the-mas-was-adjacent-to-a-coastline-noted-in-ajwtr-and-the-standard-errors.}}

\begin{Shaded}
\begin{Highlighting}[]
\FunctionTok{library}\NormalTok{(dplyr)}
\end{Highlighting}
\end{Shaded}

\begin{verbatim}
## 
## 载入程辑包:'dplyr'
\end{verbatim}

\begin{verbatim}
## The following objects are masked from 'package:stats':
## 
##     filter, lag
\end{verbatim}

\begin{verbatim}
## The following objects are masked from 'package:base':
## 
##     intersect, setdiff, setequal, union
\end{verbatim}

\begin{Shaded}
\begin{Highlighting}[]
\NormalTok{hprice}\SpecialCharTok{\%\textgreater{}\%}
  \FunctionTok{group\_by}\NormalTok{(ajwtr)}\SpecialCharTok{\%\textgreater{}\%}
  \FunctionTok{summarize}\NormalTok{(}\AttributeTok{m=}\FunctionTok{mean}\NormalTok{(hprice}\SpecialCharTok{$}\NormalTok{homeprice))}
\end{Highlighting}
\end{Shaded}

\begin{verbatim}
## # A tibble: 2 x 2
##   ajwtr      m
##   <fct>  <dbl>
## 1 0     94411.
## 2 1     94411.
\end{verbatim}

\begin{Shaded}
\begin{Highlighting}[]
\CommentTok{\# standard errors?    In group 1: \textasciigrave{}ajwtr = 0\textasciigrave{}Caused by error in \textasciigrave{}std.error()\textasciigrave{}: could not find function "std.error"}
\end{Highlighting}
\end{Shaded}

\hypertarget{test-the-difference-in-homeprice-between-coastline-msas-and-non-coastline-msas.-clearly-state-the-formula-for-the-hypothesis-the-test-method-and-your-rationale-for-selecting-the-method.-what-do-you-conclude-about-the-hypothesis}{%
\subsection{4) Test the difference in homeprice between coastline MSAs
and non-coastline MSAs. Clearly state the formula for the hypothesis,
the test method and your rationale for selecting the method. What do you
conclude about the
hypothesis?}\label{test-the-difference-in-homeprice-between-coastline-msas-and-non-coastline-msas.-clearly-state-the-formula-for-the-hypothesis-the-test-method-and-your-rationale-for-selecting-the-method.-what-do-you-conclude-about-the-hypothesis}}

\begin{Shaded}
\begin{Highlighting}[]
\CommentTok{\#Test equal variance}
\FunctionTok{var.test}\NormalTok{(homeprice }\SpecialCharTok{\textasciitilde{}}\NormalTok{ ajwtr, hprice, }\AttributeTok{alternative =} \StringTok{"two.sided"}\NormalTok{)}
\end{Highlighting}
\end{Shaded}

\begin{verbatim}
## 
##  F test to compare two variances
## 
## data:  homeprice by ajwtr
## F = 0.097496, num df = 188, denom df = 134, p-value < 2.2e-16
## alternative hypothesis: true ratio of variances is not equal to 1
## 95 percent confidence interval:
##  0.07088604 0.13297389
## sample estimates:
## ratio of variances 
##         0.09749617
\end{verbatim}

\begin{itemize}
\tightlist
\item
  We conclude that \(H_0:\sigma^2_{ajwtr=1}=\sigma^2_{ajwtr=0}\) does
  not hold\\
\end{itemize}

\begin{Shaded}
\begin{Highlighting}[]
\CommentTok{\#So we choose t{-}test}
\FunctionTok{t.test}\NormalTok{(homeprice }\SpecialCharTok{\textasciitilde{}}\NormalTok{ ajwtr, hprice, }\AttributeTok{var.equal=}\ConstantTok{FALSE}\NormalTok{, }\AttributeTok{conf.int =} \FloatTok{0.95}\NormalTok{)}
\end{Highlighting}
\end{Shaded}

\begin{verbatim}
## 
##  Welch Two Sample t-test
## 
## data:  homeprice by ajwtr
## t = -5.9922, df = 152.79, p-value = 1.43e-08
## alternative hypothesis: true difference in means between group 0 and group 1 is not equal to 0
## 95 percent confidence interval:
##  -38367.19 -19340.96
## sample estimates:
## mean in group 0 mean in group 1 
##        82388.89       111242.96
\end{verbatim}

** We reject \(H_0: \mu_{=1}=\mu_{=0}\), meaning that the homeprice is
not the same for those living on different coastlies.

\hypertarget{estimate-the-pearson-correlation-coefficient-between-homeprice-and-per-capita-income-of-the-msa-of-a-given-year-noted-in-ypc.}{%
\subsection{5) Estimate the Pearson correlation coefficient between
homeprice and per capita income of the MSA of a given year, noted in
ypc.}\label{estimate-the-pearson-correlation-coefficient-between-homeprice-and-per-capita-income-of-the-msa-of-a-given-year-noted-in-ypc.}}

\hypertarget{test-whether-the-correlation-coefficient-between-homeprice-and-ypc-is-0-or-not.-clearly-state-the-hypothesis-including-the-formula.-what-do-you-conclude}{%
\subsection{6) Test whether the correlation coefficient between
homeprice and ypc is 0 or not. Clearly state the hypothesis including
the formula. What do you
conclude?}\label{test-whether-the-correlation-coefficient-between-homeprice-and-ypc-is-0-or-not.-clearly-state-the-hypothesis-including-the-formula.-what-do-you-conclude}}

\hypertarget{can-you-say-that-per-capita-income-has-an-effect-on-the-home-sales-price-using-the-results-from-6-why-or-why-not}{%
\subsection{7) Can you say that per capita income has an effect on the
home sales price using the results from \#6)? Why or why
not?}\label{can-you-say-that-per-capita-income-has-an-effect-on-the-home-sales-price-using-the-results-from-6-why-or-why-not}}

\hypertarget{test-the-normality-of-homeprice.-would-this-test-result-change-your-responses-to-1-to-7-why-or-why-not}{%
\subsection{8) Test the normality of homeprice. Would this test result
change your responses to \#1) to 7)? Why or why
not?}\label{test-the-normality-of-homeprice.-would-this-test-result-change-your-responses-to-1-to-7-why-or-why-not}}

\end{document}
